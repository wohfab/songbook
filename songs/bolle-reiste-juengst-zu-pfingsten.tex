\beginsong{Bolle reiste jüngst zu Pfingsten
}[by={}, sr={}, cr={}, li={}]

\begin{verse}
Bolle \[G]reiste jüngst zu \[C]Pfingsten, nach \[D7]Pankow war sein \[G]Ziel
da verlor er seinen \[C]Jüngsten ganz \[D7]plötzlich im Ge\[G]wühl.
'Ne \[A]volle halbe Stunde hat \[A7]er nach \[D]ihm ge\[D7]spürt,
\end{verse}

\begin{chorus}
aber \[G]dennoch hat sich \[C]Bolle ganz \[D7]köstlich \[G]amü\[D7]siert. \rep{2}
\end{chorus}

\chordsoff

\begin{verse}
Zu Pankow gab's kein Essen, zu Pankow gab's kein Bier
war alles aufgefressen von fremden Leuten hier.
Nicht mal 'ne Butterstulle hat man ihm reserviert; aber. . .
\end{verse}

\begin{verse}
Auf der Schöneholzer Heide, da gab's 'ne Keilerei.
Und Bolle, gar nicht feige, war mittenmang dabei;
hat's Messer rausgerissen und fünfe massakriert, aber. . .
\end{verse}

\begin{verse}
Es fing schon an zu tagen, als er sein Heim erblickt.
Das Hemd war ohne Kragen, das Nasenbein zerknickt
das linke Auge fehlte, das rechte marmoriert, aber. . .
\end{verse}

\begin{verse}
Als er nach Haus gekommen, da ging's ihm aber schlecht;
da hat ihn seine Olle ganz mörderisch verdrescht!
'Ne volle halbe Stunde hat sie auf ihm poliert, aber . . .
\end{verse}

\endsong